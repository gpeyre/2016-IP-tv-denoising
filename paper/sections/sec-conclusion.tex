% !TEX root = ../TV-Denoising.tex



\section*{Conclusion}
\label{sec-conclusion}

In this paper, we have characterized the regions in which the solutions to the two-dimensional TV denoising problem are  geometrically stable under $L^2$ additive noise. In particular, via the minimal norm certificate, we introduced the notion of an extended support and although the support of TV regularized solutions are in general not stable, we have proved that the support instabilities are confined to a neighbourhood of the extended support. We have also provided explicit examples of the extended support in the case of indicators of convex sets. Within the low noise regime, for the indicator set of a calibrable set $C$, the support of the solutions was shown to cluster around $\partial C$. While for indicator functions of general convex sets (including convex sets for which the source condition is not satisfied), the support of the  solutions was shown to cluster around the domain $C\setminus \mathrm{int}(C_{R^*})$, where $C_{R^*}$ is the maximal calibrable set inside $C$.

%%%%
\section*{Acknowledgements}
The authors thank Matteo Novaga for enlightening discussions about calibrations and the capillary problem.

Antonin Chambolle was partially supported by the ANR, programs ANR-12-BS01-0014-01 ``GEOMETRYA'' and ANR-12-IS01-0003 ``EANOI'' (joint with FWF No.~I1148). He acknowledges the hospitality of DAMTP and Churchill College (U. Cambridge) for the year 2015-2016.
%
Vincent Duval and Clarice Poon acknowledge support from the CNRS (D\'efi Imag'in de la Mission pour l'Interdisciplinarit\'e, project CAVALIERI).
%
The work of Gabriel Peyr\'e has been supported by the European Research Council (ERC project SIGMA-Vision).
%
Clarice Poon acknowledges support from the Fondation Sciences Math\'{e}matiques de Paris. 
